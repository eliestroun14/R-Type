\documentclass[11pt,a4paper]{article}
\usepackage[utf8]{inputenc}
\usepackage[english]{babel}
\usepackage{geometry}
\usepackage{graphicx}
\usepackage{listings}
\usepackage{xcolor}
\usepackage{amssymb}
\usepackage{hyperref}
\usepackage{tabularx}
\usepackage{longtable}
\usepackage{array}
\usepackage{fancyhdr}
\usepackage{tcolorbox}

\geometry{margin=1in}

% Code listing settings
\lstset{
    language=C++,
    basicstyle=\ttfamily\small,
    keywordstyle=\color{blue}\bfseries,
    commentstyle=\color{green!60!black},
    stringstyle=\color{red},
    numbers=left,
    numberstyle=\tiny\color{gray},
    frame=single,
    breaklines=true,
    captionpos=b,
    tabsize=4
}

% Header and footer
\pagestyle{fancy}
\fancyhf{}
\rhead{R-Type Network Protocol v1.0}
\lhead{RFC-RTYPE-2025}
\cfoot{\thepage}

% Title page info
\title{\textbf{R-Type Network Protocol Specification}\\
\large Binary Protocol for UDP Communication\\
Version 1.0}
\author{R-Type Development Team}
\date{\today}

\begin{document}

\maketitle
\thispagestyle{empty}

\begin{abstract}
This document specifies the binary network protocol used for client-server communication in the R-Type multiplayer game. The protocol is designed to operate over UDP, providing efficient real-time data transmission with optional reliability mechanisms. This specification describes packet structures, data serialization formats, and communication patterns required for implementing compatible R-Type clients and servers.
\end{abstract}

\newpage
\tableofcontents
\newpage

\section{Introduction}

\subsection{Purpose}
This document defines the R-Type network protocol, a binary protocol designed for real-time multiplayer game communication. The protocol enables multiple clients to connect to a central authoritative server, exchange game state information, and synchronize gameplay events.

\subsection{Scope}
This specification covers:
\begin{itemize}
    \item Packet structure and format
    \item Data serialization and quantization techniques
    \item Message types and their payloads
    \item Reliability mechanisms over UDP
    \item Communication flows and patterns
\end{itemize}

\subsection{Requirements}
The keywords "MUST", "MUST NOT", "REQUIRED", "SHALL", "SHALL NOT", "SHOULD", "SHOULD NOT", "RECOMMENDED", "MAY", and "OPTIONAL" in this document are to be interpreted as described in RFC 2119.

\subsection{Protocol Characteristics}
\begin{itemize}
    \item \textbf{Transport Layer}: UDP (User Datagram Protocol)
    \item \textbf{Format}: Binary (network byte order - big-endian)
    \item \textbf{Maximum Packet Size}: 1200 bytes (recommended to avoid fragmentation)
    \item \textbf{Protocol Version}: 1.0
\end{itemize}

\section{General Packet Structure}

\subsection{Packet Format}

Every packet in the R-Type protocol consists of two main parts:

\begin{tcolorbox}[title=Packet Structure]
\begin{verbatim}
+-------------------------------------------+
|  Header (12 bytes)                        |
+-------------------------------------------+
|  Payload (variable, 0-1188 bytes)         |
+-------------------------------------------+
\end{verbatim}
\end{tcolorbox}

\subsection{Packet Header}

The packet header is a fixed-size structure present in all protocol messages. It provides essential metadata for packet processing, validation, and ordering.

\begin{lstlisting}[caption=Packet Header Structure]
struct PacketHeader {
    uint16_t magic;           // Protocol identifier: 0x5254 ('RT')
    uint8_t  packet_type;     // Packet type identifier
    uint8_t  flags;           // Control flags
    uint32_t sequence_number; // Monotonic sequence number
    uint32_t timestamp;       // Milliseconds since connection
};
// Total size: 12 bytes
\end{lstlisting}

\subsection{Header Fields Description}

\begin{longtable}{|l|l|p{8cm}|}
\hline
\textbf{Field} & \textbf{Size} & \textbf{Description} \\
\hline
\endhead

magic & 2 bytes & Protocol magic number (0x5254). Used for packet validation. Packets with invalid magic MUST be discarded. \\
\hline
packet\_type & 1 byte & Identifies the packet type. Valid range: 0x01-0x7F. See Section 4 for type definitions. \\
\hline
flags & 1 byte & Bitfield for packet control flags. See Section 2.4. \\
\hline
sequence\_number & 4 bytes & Monotonically increasing packet sequence number. Used for duplicate detection, ordering, and acknowledgment. \\
\hline
timestamp & 4 bytes & Timestamp in milliseconds since connection establishment. Used for latency measurement and interpolation. \\
\hline
\end{longtable}

\subsection{Packet Flags}

The flags field is an 8-bit bitfield used to specify packet processing requirements.

\begin{lstlisting}[caption=Packet Flags Definition]
enum PacketFlags {
    FLAG_RELIABLE    = 0x01,  // Requires acknowledgment
    FLAG_COMPRESSED  = 0x02,  // Payload is compressed
    FLAG_ENCRYPTED   = 0x04,  // Payload is encrypted
    FLAG_FRAGMENTED  = 0x08,  // Part of fragmented message
    FLAG_PRIORITY    = 0x10,  // High priority processing
    FLAG_RESERVED_5  = 0x20,  // Reserved for future use
    FLAG_RESERVED_6  = 0x40,  // Reserved for future use
    FLAG_RESERVED_7  = 0x80   // Reserved for future use
};
\end{lstlisting}

\subsubsection{Flag Descriptions}

\begin{itemize}
    \item \textbf{FLAG\_RELIABLE (0x01)}: When set, the receiver MUST send an acknowledgment. The sender MUST retransmit if no ACK is received within the timeout period.
    \item \textbf{FLAG\_COMPRESSED (0x02)}: Indicates the payload is compressed. The receiver MUST decompress before processing.
    \item \textbf{FLAG\_ENCRYPTED (0x04)}: Indicates the payload is encrypted. Implementation-specific.
    \item \textbf{FLAG\_FRAGMENTED (0x08)}: Indicates this packet is part of a larger fragmented message.
    \item \textbf{FLAG\_PRIORITY (0x10)}: Suggests priority processing. Implementation MAY prioritize these packets.
\end{itemize}

\section{Packet Types}

Packet types are organized into functional categories. Each category occupies a specific range of type identifiers.

\subsection{Type Ranges}

\begin{longtable}{|l|l|p{7cm}|}
\hline
\textbf{Range} & \textbf{Category} & \textbf{Description} \\
\hline
\endhead

0x01-0x0F & Connection & Session establishment and management \\
\hline
0x10-0x1F & Input & Player input and control messages \\
\hline
0x20-0x3F & World State & Game world and entity state updates \\
\hline
0x40-0x5F & Game Events & Gameplay events and actions \\
\hline
0x60-0x6F & Game Control & Game lifecycle and state transitions \\
\hline
0x70-0x7F & Protocol Control & Reliability and connection maintenance \\
\hline
\end{longtable}

\section{Connection Management Packets (0x01-0x0F)}

\subsection{CLIENT\_CONNECT (0x01)}

Sent by the client to initiate a connection to the server.

\begin{lstlisting}[caption=CLIENT\_CONNECT Packet]
struct ClientConnect {
    PacketHeader header;          // type = 0x01
    uint8_t protocol_version;     // Protocol version (current: 1)
    char player_name[32];         // UTF-8 player name (null-terminated)
    uint32_t client_id;           // Unique client identifier
};
// Total size: 49 bytes
\end{lstlisting}

\textbf{Fields:}
\begin{itemize}
    \item \texttt{protocol\_version}: Protocol version supported by the client. Current version is 1.
    \item \texttt{player\_name}: Player's display name, null-terminated UTF-8 string. Maximum 31 characters + null terminator.
    \item \texttt{client\_id}: Randomly generated unique identifier for this client instance.
\end{itemize}

\textbf{Usage:}
\begin{enumerate}
    \item Client generates unique \texttt{client\_id}
    \item Client sends CLIENT\_CONNECT to server
    \item Server responds with SERVER\_ACCEPT or SERVER\_REJECT
\end{enumerate}

\subsection{SERVER\_ACCEPT (0x02)}

Sent by the server to accept a client connection.

\begin{lstlisting}[caption=SERVER\_ACCEPT Packet]
struct ServerAccept {
    PacketHeader header;          // type = 0x02
    uint32_t assigned_player_id;  // Server-assigned player ID
    uint8_t max_players;          // Maximum players in game
    uint32_t game_instance_id;    // Current game instance ID
    uint16_t server_tick_rate;    // Server update rate (Hz)
};
// Total size: 23 bytes
\end{lstlisting}

\textbf{Fields:}
\begin{itemize}
    \item \texttt{assigned\_player\_id}: Unique player ID assigned by server. Client MUST use this ID in all subsequent communications.
    \item \texttt{max\_players}: Maximum number of players supported (typically 2 to 4).
    \item \texttt{game\_instance\_id}: Identifier for the game instance the player joined (can be useful for rejoining and future lobby management).
    \item \texttt{server\_tick\_rate}: Server simulation tick rate in Hz (typically 60).
\end{itemize}

\subsection{SERVER\_REJECT (0x03)}

Sent by the server to reject a connection attempt.

\begin{lstlisting}[caption=SERVER\_REJECT Packet]
struct ServerReject {
    PacketHeader header;          // type = 0x03
    uint8_t reason_code;          // Rejection reason code
    char reason_message[64];      // Human-readable reason
};
// Total size: 77 bytes
\end{lstlisting}

\textbf{Reason Codes:}
\begin{itemize}
    \item 0x00: Server full
    \item 0x01: Incompatible protocol version
    \item 0x02: Invalid player name
    \item 0x03: Banned client
    \item 0xFF: Generic error
    \item More to come with the network track
\end{itemize}

\subsection{CLIENT\_DISCONNECT (0x04)}

Sent by either client or server to terminate the connection.

\begin{lstlisting}[caption=CLIENT\_DISCONNECT Packet]
struct ClientDisconnect {
    PacketHeader header;          // type = 0x04, FLAG_RELIABLE
    uint32_t player_id;           // Disconnecting player ID
    uint8_t reason;               // Disconnect reason
};
// Total size: 17 bytes
\end{lstlisting}

\textbf{Reason Codes:}
\begin{itemize}
    \item 0x00: Normal disconnect
    \item 0x01: Timeout
    \item 0x02: Kicked by server
    \item 0x03: Client error
\end{itemize}

\textbf{Note:} This packet MUST have the FLAG\_RELIABLE flag set.

\subsection{HEARTBEAT (0x05)}

Sent periodically by clients to maintain the connection.

\begin{lstlisting}[caption=HEARTBEAT Packet]
struct Heartbeat {
    PacketHeader header;          // type = 0x05
    uint32_t player_id;           // Player ID
};
// Total size: 16 bytes
\end{lstlisting}

\textbf{Usage:}
\begin{itemize}
    \item Client SHOULD send every 1-2 seconds
    \item Server SHOULD disconnect clients that don't send heartbeat within 10 seconds
\end{itemize}

\section{Player Input Packets (0x10-0x1F)}

\subsection{PLAYER\_INPUT (0x10)}

Contains player input state. Sent frequently (typically every frame or when input changes).

\begin{lstlisting}[caption=PLAYER\_INPUT Packet]
struct PlayerInput {
    PacketHeader header;          // type = 0x10
    uint32_t player_id;           // Player ID
    uint16_t input_flags;         // Bitfield of input states
    int16_t aim_x;                // Optional: aim direction X
    int16_t aim_y;                // Optional: aim direction Y
};
// Total size: 24 bytes
\end{lstlisting}

\textbf{Input Flags Bitfield:}
\begin{lstlisting}
Bit 0:  MOVE_UP
Bit 1:  MOVE_DOWN
Bit 2:  MOVE_LEFT
Bit 3:  MOVE_RIGHT
Bit 4:  ACTION_SHOOT
Bit 5:  ACTION_SPECIAL
Bit 6:  ACTION_6 (Reserved)
Bit 7:  ACTION_7 (Reserved)
Bits 8-15: Reserved for more bonus actions
\end{lstlisting}

\textbf{Example - Packing Input:}
\begin{lstlisting}
uint16_t pack_input(bool up, bool down, bool left, 
                    bool right, bool shoot, bool special) {
    return (up << 0) | (down << 1) | (left << 2) | 
           (right << 3) | (shoot << 4) | (special << 5);
}
\end{lstlisting}

\section{World State Packets (0x20-0x3F)}

\subsection{WORLD\_SNAPSHOT (0x20)}

Contains the current state of all entities in the game world. Sent regularly by the server (typically 20-60 times per second).

\begin{lstlisting}[caption=Entity State Structure]
struct EntityState {
    uint32_t entity_id;           // Unique entity identifier
    uint8_t entity_type;          // Entity type code
    int16_t pos_x;                // Quantized X position
    int16_t pos_y;                // Quantized Y position
    int16_t vel_x;                // Quantized X velocity
    int16_t vel_y;                // Quantized Y velocity
    uint8_t health;               // Current health (0-255)
    uint8_t state_flags;          // Entity-specific state flags
};
// Total size: 16 bytes per entity
\end{lstlisting}

\begin{lstlisting}[caption=WORLD\_SNAPSHOT Packet]
struct WorldSnapshot {
    PacketHeader header;          // type = 0x20
    uint32_t world_tick;          // Server simulation tick
    uint16_t entity_count;        // Number of entities in this packet
    EntityState entities[];       // Variable-length array
};
// Total size: 18 + (16 * entity_count) bytes
\end{lstlisting}

\textbf{Entity Type Codes:}
\begin{itemize}
    \item 0x00: Player
    \item 0x01: Enemy (generic)
    \item 0x02: Enemy (snake pattern)
    \item 0x03: Enemy (boss)
    \item 0x10: Projectile (player)
    \item 0x11: Projectile (enemy)
    \item 0x20: Powerup
    \item 0x30: Obstacle
    \item 0x40: Background element
\end{itemize}

\textbf{Position Quantization:}

Positions are stored as 16-bit integers representing game world coordinates. The conversion from floating-point to quantized format is:

\begin{lstlisting}
// Quantize position (float to int16)
int16_t quantize_position(float pos, float world_min, 
                         float world_max) {
    float normalized = (pos - world_min) / (world_max - world_min);
    return (int16_t)(normalized * 65535.0f);
}

// Dequantize position (int16 to float)
float dequantize_position(int16_t quantized, float world_min,
                          float world_max) {
    float normalized = (float)quantized / 65535.0f;
    return world_min + normalized * (world_max - world_min);
}
\end{lstlisting}

\textbf{Recommended World Bounds:}
\begin{itemize}
    \item X: 0 to 2048 units
    \item Y: 0 to 1536 units
\end{itemize}

\subsection{ENTITY\_SPAWN (0x21)}

Notifies clients of a new entity entering the game world.

\begin{lstlisting}[caption=ENTITY\_SPAWN Packet]
struct EntitySpawn {
    PacketHeader header;          // type = 0x21, FLAG_RELIABLE
    uint32_t entity_id;           // New entity's unique ID
    uint8_t entity_type;          // Entity type code
    int16_t pos_x;                // Initial X position
    int16_t pos_y;                // Initial Y position
    uint8_t variant;              // Entity variant/subtype
    uint8_t initial_health;       // Starting health
    int16_t initial_velocity_x;   // Initial X velocity
    int16_t initial_velocity_y;   // Initial Y velocity
};
// Total size: 26 bytes
\end{lstlisting}

\textbf{Note:} This packet MUST have the FLAG\_RELIABLE flag set to ensure delivery.

\subsection{ENTITY\_DESTROY (0x22)}

Notifies clients that an entity has been destroyed or removed.

\begin{lstlisting}[caption=ENTITY\_DESTROY Packet]
struct EntityDestroy {
    PacketHeader header;          // type = 0x22, FLAG_RELIABLE
    uint32_t entity_id;           // Destroyed entity's ID
    uint8_t destroy_reason;       // Reason for destruction
    int16_t final_pos_x;          // Final X position (for effects)
    int16_t final_pos_y;          // Final Y position (for effects)
};
// Total size: 21 bytes
\end{lstlisting}

\textbf{Destroy Reason Codes:}
\begin{itemize}
    \item 0x00: Killed by player
    \item 0x01: Killed by enemy
    \item 0x02: Out of bounds
    \item 0x03: Timeout/despawn
    \item 0x04: Level transition
\end{itemize}

\subsection{ENTITY\_UPDATE (0x23)}

Updates specific attributes of an entity without sending full state.

\begin{lstlisting}[caption=ENTITY\_UPDATE Packet]
struct EntityUpdate {
    PacketHeader header;          // type = 0x23
    uint32_t entity_id;           // Entity to update
    uint8_t update_flags;         // Which fields are updated
    int16_t pos_x;                // Updated X position (if flag set)
    int16_t pos_y;                // Updated Y position (if flag set)
    uint8_t health;               // Updated health (if flag set)
    uint8_t shield;               // Updated shield (if flag set)
    uint8_t state_flags;          // Updated state (if flag set)
    int16_t velocity_x;           // Updated X velocity (if flag set)
    int16_t velocity_y;           // Updated Y velocity (if flag set)
};
// Total size: 26 bytes
\end{lstlisting}

\textbf{Update Flags:}
\begin{itemize}
    \item Bit 0: Position updated
    \item Bit 1: Health updated
    \item Bit 2: Shield updated
    \item Bit 3: State flags updated
    \item Bit 4: Velocity updated
    \item Bits 5-7: Reserved
\end{itemize}

\section{Game Event Packets (0x40-0x5F)}

\subsection{PLAYER\_HIT (0x40)}

Notifies that a player has been hit and taken damage.

\begin{lstlisting}[caption=PLAYER\_HIT Packet]
struct PlayerHit {
    PacketHeader header;          // type = 0x40, FLAG_RELIABLE
    uint32_t player_id;           // Player who was hit
    uint32_t attacker_id;         // Entity that caused damage
    uint8_t damage;               // Damage amount
    uint8_t remaining_health;     // Health after damage
    uint8_t remaining_shield;     // Shield after damage
    int16_t hit_pos_x;            // Hit location X
    int16_t hit_pos_y;            // Hit location Y
};
// Total size: 29 bytes
\end{lstlisting}

\subsection{PLAYER\_DEATH (0x41)}

Notifies that a player has died.

\begin{lstlisting}[caption=PLAYER\_DEATH Packet]
struct PlayerDeath {
    PacketHeader header;          // type = 0x41, FLAG_RELIABLE
    uint32_t player_id;           // Player who died
    uint32_t killer_id;           // Entity that killed player
    uint32_t score_before_death;  // Player's score
    int16_t death_pos_x;          // Death location X
    int16_t death_pos_y;          // Death location Y
};
// Total size: 30 bytes
\end{lstlisting}

\subsection{SCORE\_UPDATE (0x42)}

Updates a player's score.

\begin{lstlisting}[caption=SCORE\_UPDATE Packet]
struct ScoreUpdate {
    PacketHeader header;          // type = 0x42
    uint32_t player_id;           // Player whose score changed
    uint32_t new_score;           // Updated total score
    int16_t score_delta;          // Change in score (can be negative)
    uint8_t reason;               // Reason for score change
};
// Total size: 23 bytes
\end{lstlisting}

\textbf{Score Change Reasons:}
\begin{itemize}
    \item 0x00: Enemy killed
    \item 0x01: Boss killed
    \item 0x02: Powerup collected
    \item 0x03: Level completed
    \item 0x04: Bonus points
\end{itemize}

\subsection{POWERUP\_PICKUP (0x43)}

Notifies that a player collected a powerup.

\begin{lstlisting}[caption=POWERUP\_PICKUP Packet]
struct PowerupPickup {
    PacketHeader header;          // type = 0x43, FLAG_RELIABLE
    uint32_t player_id;           // Player who picked up powerup
    uint32_t powerup_id;          // Powerup entity ID
    uint8_t powerup_type;         // Type of powerup
    uint8_t duration;             // Effect duration (seconds, 0=permanent)
};
// Total size: 22 bytes
\end{lstlisting}

\textbf{Powerup Types:}
\begin{itemize}
    \item 0x00: Speed boost
    \item 0x01: Weapon upgrade
    \item 0x02: Force (R-Type signature weapon)
    \item 0x03: Shield
    \item 0x04: Extra life
    \item 0x05: Invincibility
\end{itemize}

\subsection{WEAPON\_FIRE (0x44)}

Notifies that an entity fired a weapon.

\begin{lstlisting}[caption=WEAPON\_FIRE Packet]
struct WeaponFire {
    PacketHeader header;          // type = 0x44
    uint32_t shooter_id;          // Entity that fired
    uint32_t projectile_id;       // New projectile entity ID
    int16_t origin_x;             // Fire origin X
    int16_t origin_y;             // Fire origin Y
    int16_t direction_x;          // Direction vector X (normalized*1000)
    int16_t direction_y;          // Direction vector Y (normalized*1000)
    uint8_t weapon_type;          // Weapon type fired
};
// Total size: 31 bytes
\end{lstlisting}

\textbf{Weapon Types:}
\begin{itemize}
    \item 0x00: Basic shot
    \item 0x01: Charged shot
    \item 0x02: Spread shot
    \item 0x03: Laser beam
    \item 0x04: Missile
    \item 0x05: Force shot
\end{itemize}

\section{Game Control Packets (0x60-0x6F)}

\subsection{GAME\_START (0x60)}

Notifies all clients that the game is starting.

\begin{lstlisting}[caption=GAME\_START Packet]
struct GameStart {
    PacketHeader header;          // type = 0x60, FLAG_RELIABLE
    uint32_t game_instance_id;    // Game instance identifier
    uint8_t player_count;         // Number of players
    uint32_t player_ids[4];       // Player IDs (up to 4)
    uint8_t level_id;             // Starting level
    uint8_t difficulty;           // Difficulty setting
};
// Total size: 36 bytes
\end{lstlisting}

\textbf{Difficulty Levels:}
\begin{itemize}
    \item 0x00: Easy
    \item 0x01: Normal
    \item 0x02: Hard
    \item 0x03: Insane
\end{itemize}

\subsection{GAME\_END (0x61)}

Notifies clients that the game has ended.

\begin{lstlisting}[caption=GAME\_END Packet]
struct GameEnd {
    PacketHeader header;          // type = 0x61, FLAG_RELIABLE
    uint8_t end_reason;           // Reason game ended
    uint32_t final_scores[4];     // Final scores for all players
    uint8_t winner_id;            // Winning player (if applicable)
    uint32_t play_time;           // Total game time (seconds)
};
// Total size: 34 bytes
\end{lstlisting}

\textbf{End Reasons:}
\begin{itemize}
    \item 0x00: Victory (all levels completed)
    \item 0x01: Defeat (all players dead)
    \item 0x02: Timeout
    \item 0x03: Host disconnect
    \item 0x04: Server shutdown
\end{itemize}

\subsection{LEVEL\_COMPLETE (0x62)}

Notifies clients that the current level has been completed.

\begin{lstlisting}[caption=LEVEL\_COMPLETE Packet]
struct LevelComplete {
    PacketHeader header;          // type = 0x62, FLAG_RELIABLE
    uint8_t completed_level;      // Level that was completed
    uint8_t next_level;           // Next level to load (0xFF=game end)
    uint32_t bonus_score;         // Completion bonus
    uint16_t completion_time;     // Time taken (seconds)
};
// Total size: 22 bytes
\end{lstlisting}

\subsection{LEVEL\_START (0x63)}

Notifies clients that a new level is starting.

\begin{lstlisting}[caption=LEVEL\_START Packet]
struct LevelStart {
    PacketHeader header;          // type = 0x63, FLAG_RELIABLE
    uint8_t level_id;             // Level identifier
    char level_name[32];          // Level name
    uint16_t estimated_duration;  // Estimated time (seconds)
};
// Total size: 47 bytes
\end{lstlisting}

\section{Protocol Control Packets (0x70-0x7F)}

\subsection{ACK - Acknowledgment (0x70)}

Acknowledges receipt of a reliable packet.

\begin{lstlisting}[caption=ACK Packet]
struct Acknowledgment {
    PacketHeader header;          // type = 0x70
    uint32_t acked_sequence;      // Sequence number being ACKed
    uint32_t received_timestamp;  // When packet was received
};
// Total size: 20 bytes
\end{lstlisting}

\textbf{Usage:}
\begin{enumerate}
    \item When receiving a packet with FLAG\_RELIABLE set, the receiver MUST send an ACK
    \item The sender MUST retransmit if ACK is not received within timeout period (typically 500ms-1000ms)
    \item Maximum retransmission attempts: 5
\end{enumerate}

\subsection{PING (0x71)}

Measures round-trip time to server.

\begin{lstlisting}[caption=PING Packet]
struct Ping {
    PacketHeader header;          // type = 0x71
    uint32_t client_timestamp;    // Client's current timestamp
};
// Total size: 16 bytes
\end{lstlisting}

\subsection{PONG (0x72)}

Response to PING packet.

\begin{lstlisting}[caption=PONG Packet]
struct Pong {
    PacketHeader header;          // type = 0x72
    uint32_t client_timestamp;    // Original client timestamp from PING
    uint32_t server_timestamp;    // Server's timestamp when received
};
// Total size: 20 bytes
\end{lstlisting}

\textbf{RTT Calculation:}
\begin{lstlisting}
uint32_t calculate_rtt(uint32_t ping_sent_time, 
                       uint32_t pong_received_time) {
    return pong_received_time - ping_sent_time;
}
\end{lstlisting}

\section{Reliability Mechanism}

\subsection{Overview}

While UDP is an unreliable protocol, certain game events require guaranteed delivery. Our R-Type protocol implements a selective reliability mechanism.

\subsection{Reliable Packet Handling}

\subsubsection{Sender Responsibilities}

\begin{enumerate}
    \item Set FLAG\_RELIABLE in packet header
    \item Store packet in retransmission queue
    \item Start timeout timer (typically 500ms)
    \item If ACK received, remove from queue
    \item If timeout expires, retransmit up to MAX\_RETRIES (5) times
    \item If all retries exhausted, consider connection lost
\end{enumerate}

\begin{lstlisting}[caption=Reliable Send Pseudocode]
void send_reliable(Packet packet) {
    packet.header.flags |= FLAG_RELIABLE;
    packet.header.sequence_number = next_sequence++;
    
    // Store for potential retransmission
    retransmit_queue.add(packet);
    
    // Send packet
    udp_send(packet);
    
    // Start timeout timer
    start_timer(packet.header.sequence_number, TIMEOUT_MS);
}

void on_timeout(uint32_t sequence) {
    Packet packet = retransmit_queue.get(sequence);
    if (packet.retry_count < MAX_RETRIES) {
        packet.retry_count++;
        udp_send(packet);
        start_timer(sequence, TIMEOUT_MS);
    } else {
        // Connection lost
        handle_connection_lost();
    }
}

void on_ack_received(uint32_t acked_sequence) {
    retransmit_queue.remove(acked_sequence);
}
\end{lstlisting}

\subsubsection{Receiver Responsibilities}

\begin{enumerate}
    \item Check if packet has FLAG\_RELIABLE set
    \item Send ACK immediately
    \item Check sequence number against last received
    \item Discard if duplicate (already processed)
    \item Process if new
\end{enumerate}

\begin{lstlisting}[caption=Reliable Receive Pseudocode]
void on_packet_received(Packet packet) {
    if (packet.header.flags & FLAG_RELIABLE) {
        // Send acknowledgment immediately
        send_ack(packet.header.sequence_number);
    }
    
    // Check for duplicate
    if (packet.header.sequence_number <= last_processed_sequence) {
        // Duplicate packet, discard
        return;
    }
    
    // Process new packet
    last_processed_sequence = packet.header.sequence_number;
    process_packet(packet);
}

void send_ack(uint32_t sequence) {
    Acknowledgment ack;
    ack.header.magic = 0x5254;
    ack.header.packet_type = 0x70;
    ack.header.sequence_number = next_sequence++;
    ack.acked_sequence = sequence;
    ack.received_timestamp = get_current_time_ms();
    
    udp_send(ack);
}
\end{lstlisting}

\subsection{Packets Requiring Reliability}

The following packet types MUST be sent with FLAG\_RELIABLE:

\begin{itemize}
    \item CLIENT\_DISCONNECT (0x04)
    \item ENTITY\_SPAWN (0x21)
    \item ENTITY\_DESTROY (0x22)
    \item PLAYER\_HIT (0x40)
    \item PLAYER\_DEATH (0x41)
    \item POWERUP\_PICKUP (0x43)
    \item GAME\_START (0x60)
    \item GAME\_END (0x61)
    \item LEVEL\_COMPLETE (0x62)
    \item LEVEL\_START (0x63)
\end{itemize}

\section{Data Serialization and Optimization}

\subsection{Byte Order}

All multi-byte integer fields MUST be transmitted in network byte order (big-endian). Implementations MUST convert between host and network byte order appropriately.

\begin{lstlisting}[caption=Byte Order Conversion]
#include <required.h>  // POSIX systems

// Serialization (host to network)
void serialize_header(PacketHeader* header, uint8_t* buffer) {
    uint16_t* buf16 = (uint16_t*)buffer;
    uint32_t* buf32 = (uint32_t*)(buffer + 4);
    
    buf16[0] = htons(header->magic);
    buffer[2] = header->packet_type;
    buffer[3] = header->flags;
    buf32[0] = htonl(header->sequence_number);
    buf32[1] = htonl(header->timestamp);
}

// Deserialization (network to host)
void deserialize_header(const uint8_t* buffer, PacketHeader* header) {
    const uint16_t* buf16 = (const uint16_t*)buffer;
    const uint32_t* buf32 = (const uint32_t*)(buffer + 4);
    
    header->magic = ntohs(buf16[0]);
    header->packet_type = buffer[2];
    header->flags = buffer[3];
    header->sequence_number = ntohl(buf32[0]);
    header->timestamp = ntohl(buf32[1]);
}
\end{lstlisting}

\subsection{Position Quantization}

To reduce bandwidth, floating-point positions are quantized to 16-bit integers.

\begin{lstlisting}[caption=Position Quantization Implementation]
// Configuration
const float WORLD_MIN_X = 0.0f;
const float WORLD_MAX_X = 2048.0f;
const float WORLD_MIN_Y = 0.0f;
const float WORLD_MAX_Y = 1536.0f;

// Clamp helper function
float clamp(float value, float min, float max) {
    if (value < min) return min;
    if (value > max) return max;
    return value;
}

// Quantize float position to int16
int16_t quantize_position_x(float x) {
    float normalized = (x - WORLD_MIN_X) / (WORLD_MAX_X - WORLD_MIN_X);
    normalized = clamp(normalized, 0.0f, 1.0f);
    return (int16_t)(normalized * 65535.0f);
}

int16_t quantize_position_y(float y) {
    float normalized = (y - WORLD_MIN_Y) / (WORLD_MAX_Y - WORLD_MIN_Y);
    normalized = clamp(normalized, 0.0f, 1.0f);
    return (int16_t)(normalized * 65535.0f);
}

// Dequantize int16 to float position
float dequantize_position_x(int16_t quantized) {
    float normalized = (float)quantized / 65535.0f;
    return WORLD_MIN_X + normalized * (WORLD_MAX_X - WORLD_MIN_X);
}

float dequantize_position_y(int16_t quantized) {
    float normalized = (float)quantized / 65535.0f;
    return WORLD_MIN_Y + normalized * (WORLD_MAX_Y - WORLD_MIN_Y);
}
\end{lstlisting}

\subsection{Velocity Quantization}

Velocities are quantized to signed 16-bit integers.

\begin{lstlisting}[caption=Velocity Quantization]
const float MAX_VELOCITY = 500.0f;  // Game units per second

int16_t quantize_velocity(float velocity) {
    float normalized = velocity / MAX_VELOCITY;
    normalized = clamp(normalized, -1.0f, 1.0f);
    return (int16_t)(normalized * 32767.0f);
}

float dequantize_velocity(int16_t quantized) {
    float normalized = (float)quantized / 32767.0f;
    return normalized * MAX_VELOCITY;
}
\end{lstlisting}

\subsection{Direction Vectors}

Normalized direction vectors are stored as 16-bit fixed-point values.

\begin{lstlisting}[caption=Direction Vector Quantization]
struct Direction {
    int16_t x;  // Range: -1000 to 1000 (represents -1.0 to 1.0)
    int16_t y;
};

Direction quantize_direction(float dx, float dy) {
    // Normalize the vector
    float length = sqrt(dx * dx + dy * dy);
    if (length > 0.0001f) {
        dx /= length;
        dy /= length;
    }
    
    Direction dir;
    dir.x = (int16_t)(dx * 1000.0f);
    dir.y = (int16_t)(dy * 1000.0f);
    return dir;
}

void dequantize_direction(Direction dir, float* dx, float* dy) {
    *dx = (float)dir.x / 1000.0f;
    *dy = (float)dir.y / 1000.0f;
}
\end{lstlisting}

\section{Communication Flows}

\subsection{Connection Establishment}

\begin{verbatim}
CLIENT                          SERVER
   |                               |
   |---(0x01) CLIENT_CONNECT------>|
   |                               |
   |                          [Validate connection]
   |                               |
   |<--(0x02) SERVER_ACCEPT--------|
   |                               |
   |                          Connection established
   |                               |
   |---(0x05) HEARTBEAT----------->|
   |                               |
   |<--(0x20) WORLD_SNAPSHOT-------|
   |                               |
\end{verbatim}

\subsection{Rejection Flow}

\begin{verbatim}
CLIENT                          SERVER
   |                               |
   |---(0x01) CLIENT_CONNECT------>|
   |                               |
   |                          [Reject: server full]
   |                               |
   |<--(0x03) SERVER_REJECT--------|
   |                               |
   |                          Connection refused
\end{verbatim}

\subsection{Normal Gameplay Loop}

\begin{verbatim}
CLIENT                          SERVER
   |                               |
   |---(0x10) PLAYER_INPUT-------->|
   |                               |
   |                          [Process input]
   |                          [Update game state]
   |                               |
   |<--(0x20) WORLD_SNAPSHOT-------|
   |                               |
   |                          [Render frame]
   |                               |
   |---(0x10) PLAYER_INPUT-------->|
   |                               |
   |<--(0x21) ENTITY_SPAWN---------|
   |---(0x70) ACK----------------->|
   |                               |
   |<--(0x20) WORLD_SNAPSHOT-------|
   |                               |
   |---(0x10) PLAYER_INPUT-------->|
   |                               |
   |<--(0x44) WEAPON_FIRE----------|
   |<--(0x20) WORLD_SNAPSHOT-------|
   |                               |
   (continues...)
\end{verbatim}

\subsection{Entity Destruction Flow}

\begin{verbatim}
CLIENT                          SERVER
   |                               |
   |---(0x10) PLAYER_INPUT-------->|
   |        [SHOOT pressed]        |
   |                               |
   |                          [Spawn projectile]
   |                               |
   |<--(0x44) WEAPON_FIRE----------|
   |<--(0x21) ENTITY_SPAWN---------|
   |---(0x70) ACK----------------->|
   |---(0x70) ACK----------------->|
   |                               |
   |<--(0x20) WORLD_SNAPSHOT-------|
   |                               |
   |                          [Collision detected]
   |                          [Enemy destroyed]
   |                               |
   |<--(0x22) ENTITY_DESTROY-------|
   |<--(0x42) SCORE_UPDATE---------|
   |---(0x70) ACK----------------->|
   |---(0x70) ACK----------------->|
   |                               |
\end{verbatim}

\subsection{Player Death and Respawn}

\begin{verbatim}
CLIENT                          SERVER
   |                               |
   |                          [Player takes damage]
   |                               |
   |<--(0x40) PLAYER_HIT-----------|
   |---(0x70) ACK----------------->|
   |                               |
   |                          [Health reaches 0]
   |                               |
   |<--(0x41) PLAYER_DEATH---------|
   |---(0x70) ACK----------------->|
   |                               |
   |                          [Wait respawn timer]
   |                               |
   |<--(0x21) ENTITY_SPAWN---------|
   |        [Player respawns]      |
   |---(0x70) ACK----------------->|
   |                               |
   |<--(0x20) WORLD_SNAPSHOT-------|
   |                               |
\end{verbatim}

\subsection{Graceful Disconnection}

\begin{verbatim}
CLIENT                          SERVER
   |                               |
   |---(0x04) CLIENT_DISCONNECT--->|
   |        [FLAG_RELIABLE]        |
   |                               |
   |                          [Remove player]
   |                          [Notify others]
   |                               |
   |<--(0x70) ACK------------------|
   |                               |
   |                          Connection closed
\end{verbatim}

\section{Security Considerations}

\subsection{Input Validation}

All received packets MUST be validated before processing:

\begin{lstlisting}[caption=Packet Validation]
bool validate_packet(const uint8_t* buffer, size_t length) {
    // Minimum size check
    if (length < sizeof(PacketHeader)) {
        return false;
    }
    
    PacketHeader header;
    deserialize_header(buffer, &header);
    
    // Magic number validation
    if (header.magic != 0x5254) {
        return false;
    }
    
    // Packet type validation
    if (header.packet_type < 0x01 || header.packet_type > 0x7F) {
        return false;
    }
    
    // Size validation based on packet type
    size_t expected_size = get_expected_packet_size(header.packet_type);
    if (length < expected_size) {
        return false;
    }
    
    return true;
}
\end{lstlisting}

\subsection{Rate Limiting}

Servers SHOULD implement rate limiting to prevent abuse:

\begin{itemize}
    \item Maximum packets per second per client: 120
    \item Maximum connection attempts per IP per minute: 10
    \item Maximum reliable packet retransmissions: 5
\end{itemize}

\subsection{Sequence Number Validation}

\begin{lstlisting}[caption=Sequence Number Validation]
bool is_sequence_valid(uint32_t received_seq, uint32_t last_seq) {
    // Allow for some reordering (window of 100)
    const uint32_t MAX_SEQUENCE_WINDOW = 100;
    
    // Handle sequence number wraparound
    if (received_seq < last_seq) {
        // Check if it's a wraparound or an old packet
        uint32_t diff = last_seq - received_seq;
        return diff > (UINT32_MAX - MAX_SEQUENCE_WINDOW);
    }
    
    // Forward sequence check
    uint32_t diff = received_seq - last_seq;
    return diff <= MAX_SEQUENCE_WINDOW;
}
\end{lstlisting}

\subsection{Buffer Overflow Prevention}

\begin{lstlisting}[caption=Safe String Copy]
void safe_string_copy(char* dest, const char* src, size_t dest_size) {
    if (dest_size == 0) return;
    
    size_t i;
    for (i = 0; i < dest_size - 1 && src[i] != '\0'; i++) {
        dest[i] = src[i];
    }
    dest[i] = '\0';
}
\end{lstlisting}

\subsection{Denial of Service Prevention}

Servers MUST implement the following protections:

\begin{itemize}
    \item Limit maximum packet size (1200 bytes)
    \item Timeout idle connections (10 seconds without heartbeat)
    \item Limit maximum entities per snapshot (64)
    \item Validate all array bounds
    \item Implement connection backlog limits
\end{itemize}

\section{Performance Optimization}

\subsection{Snapshot Optimization}

To reduce bandwidth usage for WORLD\_SNAPSHOT packets:

\begin{lstlisting}[caption=Delta Compression]
// Only send entities that changed since last snapshot
struct DeltaSnapshot {
    PacketHeader header;
    uint32_t base_tick;           // Reference tick
    uint16_t changed_entity_count;
    EntityState changed_entities[];
};
\end{lstlisting}

\subsection{Update Frequency Recommendations}

\begin{itemize}
    \item \textbf{PLAYER\_INPUT}: Every frame or on change (30-60 Hz)
    \item \textbf{WORLD\_SNAPSHOT}: 20-30 Hz (reduced from server tick rate)
    \item \textbf{HEARTBEAT}: 0.5-1 Hz
    \item \textbf{PING/PONG}: 1 Hz
\end{itemize}

\subsection{Packet Batching}

Multiple small packets MAY be combined into a single UDP datagram:

\begin{lstlisting}[caption=Packet Batching]
struct BatchedPacket {
    uint16_t magic;           // 0x5254
    uint16_t packet_count;    // Number of packets in batch
    // Followed by multiple packets with size prefix
    // [uint16_t size][packet data][uint16_t size][packet data]...
};
\end{lstlisting}

\section{Error Handling}

\subsection{Malformed Packets}

Receivers MUST handle malformed packets gracefully:

\begin{itemize}
    \item Invalid magic number: Discard silently
    \item Invalid packet type: Discard and log warning
    \item Invalid size: Discard and log warning
    \item Corrupted data: Discard and log error
\end{itemize}

\subsection{Connection Loss Detection}

\begin{lstlisting}[caption=Connection Loss Detection]
void check_connection_timeout() {
    uint32_t current_time = get_current_time_ms();
    uint32_t time_since_last_packet = current_time - last_packet_time;
    
    if (time_since_last_packet > CONNECTION_TIMEOUT_MS) {
        // No packets received for too long
        handle_connection_lost();
    }
}

const uint32_t CONNECTION_TIMEOUT_MS = 10000;  // 10 seconds
\end{lstlisting}

\subsection{Network Congestion Handling}

When network congestion is detected (high packet loss, increased latency):

\begin{itemize}
    \item Reduce WORLD\_SNAPSHOT frequency
    \item Increase client-side prediction
    \item Prioritize critical packets (player death, spawns)
    \item Reduce entity count in snapshots
\end{itemize}

\section{Implementation Guidelines}

\subsection{Creating New Packet Types}

When adding new packet types, follow these guidelines:

\begin{enumerate}
    \item Choose an appropriate type code in the correct range
    \item Define the structure with proper alignment
    \item Document all fields and their valid ranges
    \item Implement serialization/deserialization functions
    \item Add validation logic
    \item Determine if reliability is needed
    \item Update protocol version if breaking change
\end{enumerate}

\begin{lstlisting}[caption=New Packet Type Template]
// 1. Define the packet structure
struct NewPacketType {
    PacketHeader header;          // Always include header
    uint32_t field1;              // Document each field
    uint16_t field2;
    uint8_t field3;
    // ... additional fields
};

// 2. Implement serialization
void serialize_new_packet(const NewPacketType* packet, 
                          uint8_t* buffer) {
    // Serialize header
    serialize_header(&packet->header, buffer);
    
    // Serialize payload in network byte order
    uint32_t* buf32 = (uint32_t*)(buffer + 12);
    uint16_t* buf16 = (uint16_t*)(buffer + 16);
    
    buf32[0] = htonl(packet->field1);
    buf16[0] = htons(packet->field2);
    buffer[18] = packet->field3;
}

// 3. Implement deserialization
void deserialize_new_packet(const uint8_t* buffer,
                            NewPacketType* packet) {
    // Deserialize header
    deserialize_header(buffer, &packet->header);
    
    // Deserialize payload from network byte order
    const uint32_t* buf32 = (const uint32_t*)(buffer + 12);
    const uint16_t* buf16 = (const uint16_t*)(buffer + 16);
    
    packet->field1 = ntohl(buf32[0]);
    packet->field2 = ntohs(buf16[0]);
    packet->field3 = buffer[18];
}

// 4. Implement validation
bool validate_new_packet(const NewPacketType* packet) {
    // Validate field ranges
    if (packet->field3 > MAX_VALID_VALUE) {
        return false;
    }
    return true;
}
\end{lstlisting}

\subsection{Serialization Best Practices}

\begin{itemize}
    \item Always use network byte order (big-endian)
    \item Pack structures tightly (no padding)
    \item Use fixed-size integer types (uint8\_t, uint16\_t, uint32\_t)
    \item Avoid floating-point types in wire format
    \item Document byte offsets for all fields
    \item Consider alignment requirements
\end{itemize}

\subsection{Testing Recommendations}

\begin{itemize}
    \item Test packet serialization/deserialization round-trips
    \item Simulate packet loss (random drop)
    \item Simulate packet reordering
    \item Simulate packet duplication
    \item Test with high latency (250+ ms)
    \item Test with bandwidth constraints
    \item Fuzz test with malformed packets
    \item Test sequence number wraparound
\end{itemize}

\section{Appendix A: Quick Reference}

\subsection{Packet Type Summary}

\begin{longtable}{|l|l|l|l|}
\hline
\textbf{Code} & \textbf{Name} & \textbf{Reliable} & \textbf{Direction} \\
\hline
\endfirsthead

\multicolumn{4}{c}%
{{\bfseries \tablename\ \thetable{} -- continued from previous page}} \\
\hline
\textbf{Code} & \textbf{Name} & \textbf{Reliable} & \textbf{Direction} \\
\hline
\endhead

\hline \multicolumn{4}{|r|}{{Continued on next page}} \\ \hline
\endfoot

\hline
\endlastfoot

0x01 & CLIENT\_CONNECT & No & C$\rightarrow$S \\
0x02 & SERVER\_ACCEPT & No & S$\rightarrow$C \\
0x03 & SERVER\_REJECT & No & S$\rightarrow$C \\
0x04 & CLIENT\_DISCONNECT & Yes & C$\leftrightarrow$S \\
0x05 & HEARTBEAT & No & C$\rightarrow$S \\
\hline
0x10 & PLAYER\_INPUT & No & C$\rightarrow$S \\
\hline
0x20 & WORLD\_SNAPSHOT & No & S$\rightarrow$C \\
0x21 & ENTITY\_SPAWN & Yes & S$\rightarrow$C \\
0x22 & ENTITY\_DESTROY & Yes & S$\rightarrow$C \\
0x23 & ENTITY\_UPDATE & No & S$\rightarrow$C \\
\hline
0x40 & PLAYER\_HIT & Yes & S$\rightarrow$C \\
0x41 & PLAYER\_DEATH & Yes & S$\rightarrow$C \\
0x42 & SCORE\_UPDATE & No & S$\rightarrow$C \\
0x43 & POWERUP\_PICKUP & Yes & S$\rightarrow$C \\
0x44 & WEAPON\_FIRE & No & S$\rightarrow$C \\
\hline
0x60 & GAME\_START & Yes & S$\rightarrow$C \\
0x61 & GAME\_END & Yes & S$\rightarrow$C \\
0x62 & LEVEL\_COMPLETE & Yes & S$\rightarrow$C \\
0x63 & LEVEL\_START & Yes & S$\rightarrow$C \\
\hline
0x70 & ACK & No & C$\leftrightarrow$S \\
0x71 & PING & No & C$\rightarrow$S \\
0x72 & PONG & No & S$\rightarrow$C \\
\hline
\end{longtable}

\subsection{Common Values}

\begin{itemize}
    \item \textbf{Magic Number}: 0x5254 ('RT')
    \item \textbf{Protocol Version}: 1
    \item \textbf{Default Port}: 4242 (UDP)
    \item \textbf{Max Packet Size}: 1200 bytes
    \item \textbf{Connection Timeout}: 10000 ms
    \item \textbf{Heartbeat Interval}: 1000 ms
    \item \textbf{ACK Timeout}: 500-1000 ms
    \item \textbf{Max Retries}: 5
\end{itemize}

\subsection{Entity Type Codes}

\begin{itemize}
    \item 0x00: Player
    \item 0x01-0x0F: Enemies (various types)
    \item 0x10-0x1F: Projectiles
    \item 0x20-0x2F: Powerups
    \item 0x30-0x3F: Obstacles
    \item 0x40-0x4F: Background elements
\end{itemize}

\section{Appendix B: Example Implementation}

\subsection{Complete Client Connection Example}

\begin{lstlisting}[caption=Client Connection Implementation]
#include <required.h>
#include <required.h>
#include <required.h>
#include <required.h>

class RTypeClient {
private:
    int sockfd;
    struct sockaddr_in server_addr;
    uint32_t sequence_number;
    uint32_t player_id;
    bool connected;
    
public:
    RTypeClient() : sockfd(-1), sequence_number(0), 
                    player_id(0), connected(false) {}
    
    bool connect(const char* server_ip, uint16_t port) {
        // Create UDP socket
        sockfd = socket(AF_INET, SOCK_DGRAM, 0);
        if (sockfd < 0) {
            return false;
        }
        
        // Setup server address
        memset(&server_addr, 0, sizeof(server_addr));
        server_addr.sin_family = AF_INET;
        server_addr.sin_port = htons(port);
        inet_pton(AF_INET, server_ip, &server_addr.sin_addr);
        
        // Send CLIENT_CONNECT
        ClientConnect connect_packet;
        connect_packet.header.magic = 0x5254;
        connect_packet.header.packet_type = 0x01;
        connect_packet.header.flags = 0;
        connect_packet.header.sequence_number = sequence_number++;
        connect_packet.header.timestamp = get_current_time_ms();
        connect_packet.protocol_version = 1;
        strncpy(connect_packet.player_name, "Player1", 31);
        connect_packet.client_id = generate_client_id();
        
        // Serialize and send
        uint8_t buffer[256];
        serialize_client_connect(&connect_packet, buffer);
        sendto(sockfd, buffer, sizeof(ClientConnect), 0,
               (struct sockaddr*)&server_addr, sizeof(server_addr));
        
        // Wait for SERVER_ACCEPT
        if (wait_for_accept()) {
            connected = true;
            return true;
        }
        
        return false;
    }
    
    void send_input(uint16_t input_flags) {
        if (!connected) return;
        
        PlayerInput input_packet;
        input_packet.header.magic = 0x5254;
        input_packet.header.packet_type = 0x10;
        input_packet.header.flags = 0;
        input_packet.header.sequence_number = sequence_number++;
        input_packet.header.timestamp = get_current_time_ms();
        input_packet.player_id = player_id;
        input_packet.input_flags = input_flags;
        input_packet.aim_x = 0;
        input_packet.aim_y = 0;
        
        uint8_t buffer[256];
        serialize_player_input(&input_packet, buffer);
        sendto(sockfd, buffer, sizeof(PlayerInput), 0,
               (struct sockaddr*)&server_addr, sizeof(server_addr));
    }
    
    void disconnect() {
        if (!connected) return;
        
        ClientDisconnect disconnect_packet;
        disconnect_packet.header.magic = 0x5254;
        disconnect_packet.header.packet_type = 0x04;
        disconnect_packet.header.flags = FLAG_RELIABLE;
        disconnect_packet.header.sequence_number = sequence_number++;
        disconnect_packet.header.timestamp = get_current_time_ms();
        disconnect_packet.player_id = player_id;
        disconnect_packet.reason = 0x00; // Normal disconnect
        
        uint8_t buffer[256];
        serialize_client_disconnect(&disconnect_packet, buffer);
        sendto(sockfd, buffer, sizeof(ClientDisconnect), 0,
               (struct sockaddr*)&server_addr, sizeof(server_addr));
        
        connected = false;
        close(sockfd);
    }
};
\end{lstlisting}

\subsection{Complete Server Example}

\begin{lstlisting}[caption=Server Implementation Skeleton]
#include <required.h>
#include <required.h>
#include <required.h>

class RTypeServer {
private:
    int sockfd;
    std::map<uint32_t, ClientInfo> clients;
    uint32_t next_player_id;
    uint32_t sequence_number;
    
public:
    RTypeServer() : sockfd(-1), next_player_id(1), 
                    sequence_number(0) {}
    
    bool start(uint16_t port) {
        sockfd = socket(AF_INET, SOCK_DGRAM, 0);
        if (sockfd < 0) return false;
        
        struct sockaddr_in server_addr;
        memset(&server_addr, 0, sizeof(server_addr));
        server_addr.sin_family = AF_INET;
        server_addr.sin_addr.s_addr = INADDR_ANY;
        server_addr.sin_port = htons(port);
        
        if (bind(sockfd, (struct sockaddr*)&server_addr, 
                 sizeof(server_addr)) < 0) {
            return false;
        }
        
        return true;
    }
    
    void run() {
        uint8_t buffer[1200];
        struct sockaddr_in client_addr;
        socklen_t addr_len = sizeof(client_addr);
        
        while (true) {
            ssize_t received = recvfrom(sockfd, buffer, sizeof(buffer), 
                                       0, (struct sockaddr*)&client_addr, 
                                       &addr_len);
            
            if (received < 0) continue;
            
            // Validate packet
            if (!validate_packet(buffer, received)) continue;
            
            // Process packet
            PacketHeader header;
            deserialize_header(buffer, &header);
            
            switch (header.packet_type) {
                case 0x01: // CLIENT_CONNECT
                    handle_client_connect(buffer, &client_addr);
                    break;
                case 0x10: // PLAYER_INPUT
                    handle_player_input(buffer);
                    break;
                case 0x04: // CLIENT_DISCONNECT
                    handle_client_disconnect(buffer);
                    break;
                // ... other packet types
            }
        }
    }
    
    void handle_client_connect(const uint8_t* buffer, 
                               struct sockaddr_in* addr) {
        ClientConnect packet;
        deserialize_client_connect(buffer, &packet);
        
        // Validate and accept
        ServerAccept accept;
        accept.header.magic = 0x5254;
        accept.header.packet_type = 0x02;
        accept.header.flags = 0;
        accept.header.sequence_number = sequence_number++;
        accept.header.timestamp = get_current_time_ms();
        accept.assigned_player_id = next_player_id++;
        accept.max_players = 4;
        accept.game_instance_id = 1;
        accept.server_tick_rate = 60;
        
        uint8_t send_buffer[256];
        serialize_server_accept(&accept, send_buffer);
        sendto(sockfd, send_buffer, sizeof(ServerAccept), 0,
               (struct sockaddr*)addr, sizeof(*addr));
    }
};
\end{lstlisting}

\section{Appendix C: Packet Size Reference}

\begin{longtable}{|l|l|p{6cm}|}
\hline
\textbf{Packet Type} & \textbf{Size (bytes)} & \textbf{Notes} \\
\hline
\endhead

CLIENT\_CONNECT & 49 & Fixed size \\
\hline
SERVER\_ACCEPT & 23 & Fixed size \\
\hline
SERVER\_REJECT & 77 & Fixed size \\
\hline
CLIENT\_DISCONNECT & 17 & Fixed size \\
\hline
HEARTBEAT & 16 & Fixed size \\
\hline
PLAYER\_INPUT & 24 & Fixed size \\
\hline
WORLD\_SNAPSHOT & 18 + 16n & Variable, n = entity count \\
\hline
ENTITY\_SPAWN & 26 & Fixed size \\
\hline
ENTITY\_DESTROY & 21 & Fixed size \\
\hline
ENTITY\_UPDATE & 26 & Fixed size \\
\hline
PLAYER\_HIT & 29 & Fixed size \\
\hline
PLAYER\_DEATH & 30 & Fixed size \\
\hline
SCORE\_UPDATE & 23 & Fixed size \\
\hline
POWERUP\_PICKUP & 22 & Fixed size \\
\hline
WEAPON\_FIRE & 31 & Fixed size \\
\hline
GAME\_START & 36 & Fixed size \\
\hline
GAME\_END & 34 & Fixed size \\
\hline
LEVEL\_COMPLETE & 22 & Fixed size \\
\hline
LEVEL\_START & 47 & Fixed size \\
\hline
ACK & 20 & Fixed size \\
\hline
PING & 16 & Fixed size \\
\hline
PONG & 20 & Fixed size \\
\hline
\end{longtable}

\section{Appendix D: Bandwidth Calculations}

\subsection{Typical Bandwidth Usage}

Assuming a 4-player game with typical update frequencies:

\textbf{Client Upload (per client):}
\begin{itemize}
    \item PLAYER\_INPUT: 24 bytes × 60 Hz = 1,440 bytes/s
    \item HEARTBEAT: 16 bytes × 1 Hz = 16 bytes/s
    \item PING: 16 bytes × 1 Hz = 16 bytes/s
    \item \textbf{Total Upload:} $\sim$1.5 KB/s = 12 Kbps
\end{itemize}

\textbf{Server Broadcast (per client):}
\begin{itemize}
    \item WORLD\_SNAPSHOT (40 entities): (18 + 16×40) bytes × 30 Hz = 19,740 bytes/s
    \item ENTITY\_SPAWN: 26 bytes × 10/s = 260 bytes/s (average)
    \item ENTITY\_DESTROY: 21 bytes × 10/s = 210 bytes/s (average)
    \item WEAPON\_FIRE: 31 bytes × 20/s = 620 bytes/s (average)
    \item PONG: 20 bytes × 1 Hz = 20 bytes/s
    \item ACKs: Variable, $\sim$200 bytes/s
    \item \textbf{Total Download:} $\sim$21 KB/s = 168 Kbps
\end{itemize}

\textbf{Server Total Bandwidth (4 clients):}
\begin{itemize}
    \item Upload: 21 KB/s × 4 = 84 KB/s = 672 Kbps
    \item Download: 1.5 KB/s × 4 = 6 KB/s = 48 Kbps
    \item \textbf{Total Server:} $\sim$90 KB/s = 720 Kbps
\end{itemize}

\subsection{Optimization Strategies}

To reduce bandwidth when needed:

\begin{enumerate}
    \item \textbf{Reduce snapshot frequency:} 30 Hz → 20 Hz saves 33\% on WORLD\_SNAPSHOT
    \item \textbf{Send only visible entities:} Culling off-screen entities can reduce entity count by 50-70\%
    \item \textbf{Delta compression:} Only send changed entities, potentially reducing by 80\%
    \item \textbf{Priority system:} Send closer/important entities more frequently
    \item \textbf{Quantization:} Already implemented with int16 positions
\end{enumerate}

\section{Appendix E: Error Codes Reference}

\subsection{Disconnect Reasons (0x04)}

\begin{longtable}{|l|l|}
\hline
\textbf{Code} & \textbf{Description} \\
\hline
\endhead

0x00 & Normal disconnect (user quit) \\
0x01 & Connection timeout (no heartbeat) \\
0x02 & Kicked by server (admin action) \\
0x03 & Client error (crash/exception) \\
\hline
\end{longtable}

\subsection{Server Reject Reasons (0x03)}

\begin{longtable}{|l|l|}
\hline
\textbf{Code} & \textbf{Description} \\
\hline
\endhead

0x00 & Server full (max players reached) \\
0x01 & Incompatible protocol version \\
0x02 & Invalid player name (empty, too long, invalid chars) \\
0x03 & Banned client (IP or client\_id banned) \\
0xFF & Generic error (server internal error) \\
\hline
\end{longtable}

\subsection{Entity Destroy Reasons (0x22)}

\begin{longtable}{|l|l|}
\hline
\textbf{Code} & \textbf{Description} \\
\hline
\endhead

0x00 & Killed by player projectile \\
0x01 & Killed by enemy projectile \\
0x02 & Out of bounds (left play area) \\
0x03 & Timeout/despawn (lifetime expired) \\
0x04 & Level transition (new level loading) \\
\hline
\end{longtable}

\subsection{Score Update Reasons (0x42)}

\begin{longtable}{|l|l|}
\hline
\textbf{Code} & \textbf{Description} \\
\hline
\endhead

0x00 & Enemy killed (standard) \\
0x01 & Boss killed (major bonus) \\
0x02 & Powerup collected \\
0x03 & Level completed (completion bonus) \\
0x04 & Bonus points (combo, time bonus, etc.) \\
\hline
\end{longtable}

\section{Appendix F: Debugging and Logging}

\subsection{Recommended Log Format}

For debugging network issues, implement structured logging:

\begin{lstlisting}[caption=Packet Logging Format]
void log_packet(const char* direction, const PacketHeader* header) {
    printf("[%s] Type=0x%02X Seq=%u Flags=0x%02X Time=%u\n",
           direction,
           header->packet_type,
           header->sequence_number,
           header->flags,
           header->timestamp);
}

// Usage:
log_packet("SEND", &packet.header);  // \rightarrow
log_packet("RECV", &packet.header);  // \leftarrow
\end{lstlisting}

\subsection{Network Statistics}

Track these metrics for debugging and monitoring:

\begin{itemize}
    \item \textbf{Packets sent/received:} Total count per type
    \item \textbf{Bytes sent/received:} Bandwidth usage
    \item \textbf{Packet loss rate:} Percentage of lost packets
    \item \textbf{Round-trip time (RTT):} Average and peak latency
    \item \textbf{Retransmissions:} Count of reliable packet retries
    \item \textbf{Out-of-order packets:} Sequence number gaps
    \item \textbf{Duplicate packets:} Same sequence received multiple times
\end{itemize}

\begin{lstlisting}[caption=Network Statistics Structure]
struct NetworkStats {
    uint64_t packets_sent;
    uint64_t packets_received;
    uint64_t bytes_sent;
    uint64_t bytes_received;
    uint32_t packets_lost;
    uint32_t packets_retransmitted;
    uint32_t duplicates_received;
    uint32_t out_of_order_received;
    float avg_rtt_ms;
    float peak_rtt_ms;
    
    void print_stats() {
        printf("=== Network Statistics ===\n");
        printf("Packets: Sent=%llu Recv=%llu Lost=%u (%.2f%%)\n",
               packets_sent, packets_received, packets_lost,
               (float)packets_lost / packets_sent * 100.0f);
        printf("Bytes: Sent=%llu Recv=%llu\n", 
               bytes_sent, bytes_received);
        printf("RTT: Avg=%.2fms Peak=%.2fms\n", 
               avg_rtt_ms, peak_rtt_ms);
        printf("Retransmissions: %u\n", packets_retransmitted);
    }
};
\end{lstlisting}

\section{Appendix G: Future Extensions}

\subsection{Potential Protocol Extensions}

The protocol can be extended in future versions to support:

\begin{itemize}
    \item \textbf{Voice chat:} New packet types for audio streams (0x80-0x8F range)
    \item \textbf{Lobby system:} Room management packets (0x90-0x9F range)
    \item \textbf{Replay system:} Recorded game data packets
    \item \textbf{Server browser:} Server info query packets
    \item \textbf{Anti-cheat:} Validation and integrity check packets
    \item \textbf{Spectator mode:} Observe-only connections
    \item \textbf{NAT traversal:} Peer-to-peer connection establishment
\end{itemize}

\subsection{Version Negotiation}

When introducing breaking changes, update the protocol version:

\begin{lstlisting}[caption=Version Negotiation]
// In CLIENT_CONNECT:
protocol_version = 2;  // New version

// Server checks:
if (packet.protocol_version != SUPPORTED_VERSION) {
    send_rejection(SERVER_REJECT_INCOMPATIBLE_VERSION);
}
\end{lstlisting}

For backward compatibility, the server MAY support multiple protocol versions simultaneously by maintaining version-specific packet handlers.

\section{Appendix H: Testing Checklist}

\subsection{Unit Tests}

\begin{itemize}
    \item[$\square$] Packet serialization/deserialization for all types
    \item[$\square$] Byte order conversion (endianness)
    \item[$\square$] Quantization/dequantization accuracy
    \item[$\square$] Header validation (magic, type, size)
    \item[$\square$] Sequence number handling and wraparound
    \item[$\square$] String safety (buffer overflow prevention)
\end{itemize}

\subsection{Integration Tests}

\begin{itemize}
    \item[$\square$] Client connection and disconnection
    \item[$\square$] Multiple clients simultaneously
    \item[$\square$] Reliable packet delivery and ACK
    \item[$\square$] Packet loss handling
    \item[$\square$] Out-of-order packet handling
    \item[$\square$] Duplicate packet detection
    \item[$\square$] Connection timeout and recovery
    \item[$\square$] High latency scenarios (250+ ms)
\end{itemize}

\subsection{Performance Tests}

\begin{itemize}
    \item[$\square$] Bandwidth usage under normal load
    \item[$\square$] CPU usage for packet processing
    \item[$\square$] Memory usage for packet queues
    \item[$\square$] Maximum concurrent clients
    \item[$\square$] Packet processing throughput
    \item[$\square$] Network saturation recovery
\end{itemize}

\subsection{Security Tests}

\begin{itemize}
    \item[$\square$] Malformed packet rejection
    \item[$\square$] Oversized packet handling
    \item[$\square$] Rate limiting effectiveness
    \item[$\square$] Invalid magic number handling
    \item[$\square$] Sequence number spoofing prevention
    \item[$\square$] Buffer overflow protection
\end{itemize}

\section{Conclusion}

This protocol specification provides a solid foundation for implementing the R-Type multiplayer game. The binary format ensures efficient bandwidth usage, while the reliability mechanism guarantees critical game events are delivered. The modular design allows for easy extension and maintenance as the game evolves.

Key takeaways:

\begin{itemize}
    \item \textbf{Efficiency:} Binary format with quantization minimizes bandwidth
    \item \textbf{Reliability:} Selective ACK system for critical packets
    \item \textbf{Scalability:} Supports multiple clients and game instances
    \item \textbf{Security:} Input validation and rate limiting prevent abuse
    \item \textbf{Extensibility:} Reserved packet ranges and flags for future features
\end{itemize}

Developers implementing this protocol should refer to the packet type reference (Appendix A), example implementations (Appendix B), and testing checklist (Appendix H) to ensure correct and robust implementation.

\subsection{Version History}

\begin{itemize}
    \item \textbf{Version 1.0} (2025-01-XX): Initial protocol specification
    \begin{itemize}
        \item Core packet types for connection, input, world state, and events
        \item Reliability mechanism with selective ACK
        \item Position and velocity quantization
        \item Security and validation guidelines
    \end{itemize}
\end{itemize}

\section{Appendix I: References}

\begin{itemize}
    \item RFC 768: User Datagram Protocol
    \newline \url{https://www.rfc-editor.org/rfc/rfc768}
    
    \item RFC 2119: Key words for use in RFCs to Indicate Requirement Levels
    \newline \url{https://www.rfc-editor.org/rfc/rfc2119}
    
    \item Gaffer On Games: "Networked Physics"
    \newline \url{https://gafferongames.com/post/networked_physics_2004/}
    
    \item Valve Developer Community: "Source Multiplayer Networking"
    \newline \url{https://developer.valvesoftware.com/wiki/Source_Multiplayer_Networking}
    
    \item Gabriel Gambetta: "Fast-Paced Multiplayer" (Client-Side Prediction)
    \newline \url{https://www.gabrielgambetta.com/client-side-prediction-server-reconciliation.html}
    
    \item Glenn Fiedler: "UDP vs TCP"
    \newline \url{https://gafferongames.com/post/udp_vs_tcp/}
    
    \item "Quake 3 Network Model"
    \newline \url{https://fabiensanglard.net/quake3/network.php}
\end{itemize}

\section{Appendix J: Glossary}

\begin{description}
    \item[ACK (Acknowledgment)] A confirmation message sent to indicate successful receipt of a packet.
    
    \item[Big-endian] Byte ordering where the most significant byte is stored first (network byte order).
    
    \item[Datagram] A self-contained, independent packet of data sent over UDP.
    
    \item[Dequantization] Converting compressed integer representation back to floating-point values.
    
    \item[Entity] Any game object (player, enemy, projectile, powerup, etc.).
    
    \item[Latency] The time delay between sending and receiving a packet (also called "lag").
    
    \item[Magic Number] A constant value used to validate packet format (0x5254 for R-Type).
    
    \item[Packet Loss] When network packets fail to reach their destination.
    
    \item[Quantization] Converting floating-point values to smaller integer representation to save bandwidth.
    
    \item[RTT (Round-Trip Time)] The time for a packet to travel from sender to receiver and back.
    
    \item[Sequence Number] A monotonically increasing counter used to detect packet loss and duplication.
    
    \item[Snapshot] A complete state update of the game world at a specific point in time.
    
    \item[TCP (Transmission Control Protocol)] A reliable, connection-oriented transport protocol.
    
    \item[UDP (User Datagram Protocol)] An unreliable, connectionless transport protocol optimized for speed.
    
    \item[World Tick] A discrete time step in the server's game simulation (typically 60 Hz = 16.67ms per tick).
\end{description}

\vspace{1cm}
\hrule
\vspace{0.5cm}
\begin{center}
\textbf{End of Document}

\vspace{0.5cm}
For questions, issues, or contributions to this protocol specification, \\
please contact our R-Type development team.
\end{center}

\end{document}
